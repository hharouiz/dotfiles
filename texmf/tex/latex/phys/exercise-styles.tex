% Exercice styles
\mdfdefinestyle{ExerciceStyle1}{%
    % Options with lengths
    skipabove = \baselineskip,
    skipbelow = 0pt,
    leftmargin = 0pt,
    rightmargin = 0pt,
    innerleftmargin = 6mm,
    innerrightmargin = 10pt,
    innertopmargin = 5pt,
    innerbottommargin = 5pt,
    % linewidth = 0.4pt,
    innerlinewidth = 0pt,
    middlelinewidth = 0.4pt,
    outerlinewidth = 0pt,
    roundcorner = 0pt,
    % Colored Options
    linecolor = black,
    innerlinecolor = black,
    middlelinecolor = black,
    outerlinecolor = black,
    backgroundcolor = white,
    fontcolor = black,
    % Hidden Lines
    topline = true,
    rightline = true,
    leftline = true,
    bottomline = true,
    hidealllines = true,
    % Frametitle
    frametitleaboveskip = 0pt,%
    frametitlebelowskip = 0pt,%
    % Extra
    singleextra = {%
        \stepcounter{exercice}%
        \newdimen\pr%
        \pr=5pt
        \newdimen\lw%
        \lw=0.8pt
        \newdimen\pb%
        \pb=0.6pt
        \tikzstyle{boxborderstyle}=[color=capri,line width=\lw,%
        rounded corners=\pr+0.5\lw+0.5\pb]%
        \node[rectangle,anchor=south west,inner xsep=0pt,font=\bfseries,
        text=capri] (title) at (O |- P) {Exercice~\theexercice};%
        \draw[boxborderstyle] (title.south east) -| (O) -- (O -| title.east);
        \foreach \n/\y in {O|-P/-\pr-0.5\lw-0.5\pb,O/\pr+0.5\lw+0.5\pb}{%
            \node[circle,draw=white,fill=capri,inner sep=0pt,
            minimum size=2\pr,line width=\pb] (pulley)
            at ([shift={(\pr+0.5\lw+0.5\pb,\y)}]\n) {};
            \fill[white] (pulley.center) circle (1pt);
        }
    },%
    nobreak,
}

\mdfdefinestyle{ExerciceStyle2}{%
    skipabove = \baselineskip,
    skipbelow = \baselineskip,
    innerleftmargin = 0pt,
    innerrightmargin = 0pt,
    innertopmargin = 5pt,
    innerbottommargin = 5pt,
    middlelinewidth = 0.4pt,
    %hidealllines = true,
    singleextra={
        %        \fill[primarycolor]
        %        (O |- P)  -- +(3mm,-0.5\mdfframetitleboxtotalheight) --
        %        +(0,-\mdfframetitleboxtotalheight) -- cycle;
        \node[anchor=south] at (O |- P) {Exercice~\theexercice};
    },
    nobreak
}


\definecolor{background}{HTML}{f5f3f4}
\mdfdefinestyle{ExerciceStyle3}{%
    skipabove = \baselineskip,
    skipbelow = 0pt,
    innerleftmargin = 10pt,
    innerrightmargin = 10pt,
    innertopmargin = 10pt,
    innerbottommargin = 5pt,
    hidealllines = true,
    leftline = true,
    linewidth = 2pt,
    frametitleaboveskip=5pt,
    frametitlebelowskip = 5pt,%
    frametitlefont={\normalfont\large\bfseries},
    backgroundcolor=background,
    linecolor=primarycolor,
    singleextra={
        \stepcounter{exercice}%
        \draw[gray]
            ($(O |- P)+(10pt,-\mdfframetitleboxtotalheight)$) --
            ($(P)+(0,-\mdfframetitleboxtotalheight)$);
    },
    %nobreak
}
